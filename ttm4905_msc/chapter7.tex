\chapter{Proposal of Business Model}
This chapter will present the reader with the freemium-based business model proposal for MazeMap. 


\section{Business model}
\subsection{Customer Segments}

\begin{itemize}
    \item Users of mazemap will want to be someone who resides over a certain area
    \item Freemium might be harder to implement, due to regulatory affairs.
    \item what needs to be in place customer side: low amount of financial means, prefferably wifi for navigation (might be seen as detrimental, but also may hamper the service). Existing customer base. 
\end{itemize}

\subsection{Value Propositions}

\begin{itemize}
    \item  Fleet management
    \item Reduce missed appointments
    \item For freemium: Cost saving by being initially free
    \item For freemium: Try before you buy, no big initial investment. buy after need
    \item Maintenance personell: Professions that see frequent changes of staff and working locations. Indoor maps enable these to find their way quicker, saving personell costs
    \item Directions for meeting place and nearest parking for visitors
    \item Alarms shows up on maps rather than a code for a specific location. Can save time in 
    \item Equipment tracking, Internet of things
    \item For Freemium: 
\end{itemize}

\subsection{Channels}

\begin{itemize}
    \item Cisco marketplace
    \item Webpage
    \item Develop leads inside businesses
    \item For freemium: Hard to argue against initial high costs as with a non-freemium system
\end{itemize}

\subsection{Customer Relationships}
\begin{itemize}
    \item For freemium: Guidance as a service
    \item As service matures, less active relationships, more automation
    \item Interactions between the Businesses of the B2B\&C paradigm, sharing of experiences
\end{itemize}

\subsection{Revenue Streams}
\begin{itemize}
    \item Concept of freemium: The additional services paid for by the few pays for the many customers.
    \item Additional paid services: Updating of maps, integration with timetabling, sms, existing IT systems, indoor pathing, indoor positioning, analytics, map editor.
    
\end{itemize}

\subsection{Key Activities}
\begin{itemize}
    \item Indoor mapping roughly ~80\% of usage
    \item Integration of maps
    \item Rework and update map data
    \item Accomodating for indoor navigation and pathfinding, via Cisco and other partners
    \item Integration into booking systems
    \item Marketing, sales and customer support
    \item Key Account management - large customers with many needs
    \item After purchases: Revenue through added services as seen fit by the establishment
\end{itemize}

\subsection{Key Resources}
\begin{itemize}
    \item MM servers
    \item MM's floor plan to interactive map engine
    \item software developers
    \item Customer's mapping data + metadata
\end{itemize}

\subsection{Key Partnerships}
\begin{itemize}
    \item Positional services i.e. Cisco
    \item Cloud storage: servers + redundancy nodes for better access throughout the world.
\end{itemize}

\subsection{Cost Structure}
\begin{itemize}
    \item Value driven: value creation and counceling
    \item Offices/personell: development, renting facilities.
    \item High startup costs: software development team, servers, offices
    \item Low running costs: scalable platform, customer administrates maps (can add POIs etc)
\end{itemize}