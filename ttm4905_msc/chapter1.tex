\chapter{Chapter 1 - Introduction}
This chapter will introduce the reader to this thesis, where the motivation, scope and contribution is presented, which (along with Chapter 2 - Background) will serve to bind the thesis together and to set things in perspective to the reader.
\section{Motivation}
as
\newline
%hvorfor navigasjon er viktig
\\
The indoor mapping service \gls{mm} started as a venture from Wireless Trondheim, an R\&D company working closely (and cooperating) with \gls{ntnu} whose goal it is to create sustainable ventures from new ideas. MazeMap's main service is providing an indoor mapping service aimed at large institutions such as universities, hospitals, conference venues, shopping malls and more. Users are able to access the service on any computer, tablet or smartphone, where the indoor maps themselves is the main focal point. Navigation services are also available in many different forms, but using Wi-Fi is a readily available and easily deployed solution which has been co-developed with Cisco. This particular service uses a technique called trilateration\cite{BiczokMJK14}, and is easily implemented as it uses existing Wi-Fi infrastructure to facilitate the service.
\newline
\\
\gls{mm} has a very scalable technical platform, in which a robust indoor-map-creating engine resides. This engine is able to convert digital floorplans into full-scale digital indoor maps that can be accessed from any device capable of running \gls{mm}'s application. Given its scalable nature, the time needed per customer is drastically reduced, and little involvement is needed from \gls{mm}'s perspective. Thus, in order to accelerate customer acquisition, a proposition can be made to change or alter the business model: A free model has been detrimental in the success of several start-ups in the consumer market, e.g. Dropbox, Skype, Waze and Snapchat, but can this also be applied in a \gls{b2b} or a \gls{b2bc} setting?

\subsection{asda}

\section{Scope \& Objectives}
%Customer segs: 2 different: HEIs and Hospitals
In order to achieve the goal of determining the effectiveness and feasability of a free model in an international \gls{b2bc}-market and proposing a fitting business model, it is necessary to narrow down the aspects considered in this thesis. Furthermore, the aim is also to give the reader a clear and concise overview of the matter at hand.

\subsection{Scope}
MazeMap already serves a large base of customers around the world, with their main bottleneck in expanding further being customer acquisition. As means to remedy this and to accelerate customer acquisition the free model is being proposed. Given \gls{mm}'s robust map-generating-engine, the main focus is shifted away from any technical limitations or inherent flaws on \gls{mm}'s end, and is shifted towards the viability of a free model. Given the scale of a global survey, the survey presented later in the thesis will focus on an already established customer segment, namely \gls{hei}s. This restriction is in place in order to more specifically target this thesis's goal of proposing a business model and determining if a free model is feasible, rather than exploring new customer segments. 

\subsection{Objectives}
This thesis aims to determine the viability and feasability of a free model in a \gls{b2bc}-market, and to propose an appropriate business model in this particular paradigm. In short we can describe the objectives in the following manner:
\begin{enumerate}
    \item Investigate if there is a demand for an indoor mapping service operating under the free model 
    \item Discuss and interpret the viability of such a service at an international level
    \item Propose a business model based on the findings and its surrounding discussion
\end{enumerate}
Taking the scope into consideration, this forms the basis of the main research question for this thesis: Is a free model viable as a business model in a \gls{b2bc}-market on an international level, and can this model potentially accelerate customer acquisition?

\section{Contribution}
Mainly, the contribution and novelty of this thesis is aimed at business owners wishing to expand their offerings to their base of customers, by enabling an indoor mapping service such as MazeMap. Furthermore, the central theme of the thesis regarding the viability of freemium in a \gls{b2bc} market, may also serve businesses looking to expand their offerings and who dares to venture in new and alternative business models. The key contributions consists of the market survey made to target the objectives set in 1.2.2 Objectives, and the resulting proposal of a business model based upon the data gathered.   
\section{Outline}

\section{Related Work}
Wireless Trondheim along with MazeMap have both provided several semester projects and master theses in cooperation with \gls{ntnu}.