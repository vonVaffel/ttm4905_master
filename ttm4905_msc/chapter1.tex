\chapter{Introduction}
This chapter will introduce the reader to this thesis, where the motivation, scope and contribution is presented, which (along with Chapter 2) will serve to bind the thesis together and bring perspective to the reader.
\section{Motivation}

%hvorfor navigasjon er viktig

The indoor mapping service \gls{mm} started as a venture from Wireless Trondheim, an R\&D company working closely (and cooperating) with \gls{ntnu} whose goal it is to create sustainable ventures from new ideas. MazeMap's main service is providing an indoor mapping service aimed at large institutions such as universities, hospitals, conference venues, shopping malls and more. Users are able to access the service on any computer, tablet or smartphone, where the indoor maps themselves is the main focal point. Navigation services are also available in many different forms, but using Wi-Fi is a readily available and easily deployed solution which has been co-developed with Cisco. This particular service uses a technique called trilateration\cite{BiczokMJK14}, and is easily implemented as it uses existing Wi-Fi infrastructure to facilitate the service.


\gls{mm} has a very scalable technical platform, in which a robust indoor-map-creating engine resides. This engine is able to convert digital floorplans into full-scale digital indoor maps that can be accessed from any device capable of running \gls{mm}'s application. Given its scalable nature, the time needed per customer is drastically reduced, and little involvement is needed from \gls{mm}'s perspective. Thus, in order to accelerate customer acquisition, a proposition can be made to change or alter the business model: A free model has been detrimental in the success of several start-ups in the consumer market, e.g. Dropbox, Skype, Waze and Snapchat, but it is of interest to see if this can be applied in a \gls{b2b} or a \gls{b2bc} setting with the same success.


Two market drivers are particularly protruding when it comes to indoor maps:
\begin{itemize}
    \item The demand for indoor orientation as buildings become richer in numbers and more complex.
    \item The Internet of Things and fleet management.
\end{itemize}


\section{Scope \& Objectives}
%Customer segs: 2 different: HEIs and Hospitals
In order to achieve the goal of determining the effectiveness and feasibility of a free model in an international \gls{b2bc}-market and proposing a fitting business model, it is necessary to narrow down the aspects considered in this thesis. Furthermore, the aim is also to give the reader a clear and concise overview of the matter at hand.

\subsection{Scope}
MazeMap already serves a large base of customers around the world, with their main bottleneck in expanding further being customer acquisition. As means to remedy this and to accelerate customer acquisition the freemium can be proposed. Given \gls{mm}'s robust map-generating-engine, the main focus is shifted away from any technical limitations or inherent flaws on \gls{mm}'s end, and is shifted towards the viability of a freemium model. Given the scale of a global survey, the survey presented later in the thesis will focus on an already established customer segment, namely \gls{hei}s. This restriction is in place in order to more specifically target this thesis's goal of proposing a business model and determining if a freemium business model is feasible, rather than exploring new customer segments. 

\subsection{Objectives}
This thesis aims to determine the viability and feasibility of a free model in a \gls{b2bc}-market, and to propose an appropriate business model in this particular paradigm. In short we can describe the objectives in the following manner:
\begin{enumerate}
    \item Investigate if there is a demand for an \gls{ims} operating under the free model 
    \item Discuss and interpret the viability of such a service at an international level
    \item Propose a business model based on the findings and its surrounding discussion
\end{enumerate}
Taking the scope into consideration, this forms the basis of the main research question for this thesis: Is a free model viable as a business model in a \gls{b2bc}-market on an international level, and can this model potentially accelerate customer acquisition?

\section{Contribution}
Mainly, the contribution and novelty of this thesis is aimed at business owners wishing to expand their offerings to their base of customers, by enabling an indoor mapping service such as MazeMap. Furthermore, the central theme of the thesis regarding the viability of freemium in a \gls{b2bc} market, may also serve businesses looking to expand their offerings and who dares to venture in new and alternative business models. The key contributions consists of the market survey made to answer the objectives set in the section above, and the resulting proposal of a business model based upon the data gathered.   

\section{Outline}
Chapter 1 introduces the reader to the thesis where motivation, scope and objectives are presented alongside the contributions and related work. In the second chapter, relevant terms such as "freemium", "B2B", "B2C" and "B2B\&C" are introduced in order to provide the user with definitions central to the thesis' themes. Chapter 3 presents the methodology employed for researching the problem and introduces the survey. Chapter 4 discusses various companies and their products, and how these have gained success or failed to obtain it, by employing a freemium business model in a \gls{b2b} market. Chapter 5 presents and discusses the results obtained in the survey, and chapter 6 presents the proposed, freemium-based business model based upon the research done.
Chapter 7 covers the concluding remarks along with suggestions for further studies and research on the subject. Appendix A contains the invitation letter that was sent in relation to the survey, and Appendix B presents the survey as it was presented to the respondents. Lastly Appendix C contains a description of the Business Model Canvas framework used for proposing a business model in Chapter 6. 


\section{Related Work}
Wireless Trondheim along with MazeMap have both provided several semester projects and master theses in cooperation with \gls{ntnu}. Three master theses/semester projects protrudes as being particularly relevant in relation to this thesis: "Business Potential for Data from Wi-Fi Networks" by Bergdendal, Petter 2014, "Campusguiden" by Halvorsen, Christian 2011 and "International Business Potential for Analytics of Room Utilisation" by Binde, Karl 2015~\cite{petterbergendal2014} \cite{christianhalvorsen2011} \cite{karlbernhoffbinde2015}. The first of these focuses on MazeMap with the targeted customer segment being shopping malls, and the research done was centered around a survey sent out to various shopping malls. The main goal of this semester project was to discover a way to generate value from analytics of data provided by Wi-Fi usage, with features discussed being counting visitors, duration of visits and flow of visitors. In relation to this thesis, the market segment and targeted countries differ. The second focuses on MazeMap's predecessor "Campusguiden", with technical aspects of the service being the focal points, but various financial aspects were discussed as well. Compared to this thesis, its methodology were solely qualitative in nature, and focuses more on the technical aspect more than what is discussed in this thesis. The latter of these focuses on both the technical and financial aspects, with the main emphasis being on the financial aspects. Results were obtained through surveys sent out to \glspl{hei} worldwide, eventually producing a proposed business model for MazeMap that focused on how analytics could be used to better utilise rooms for \glspl{hei}. The targeted customer segment is the same as in this thesis, and as such makes it relevant. However, the proposed business models and scope vary greatly from this thesis


The master thesis "Freemium for Large Enterprises" by Jepson, Lundin 2011~\cite{jepson2009freemium} is relevant to this thesis in that it is a case study of a \gls{b2b} based company that has successfully implemented freemium as a business model. It discusses how freemium can be employed to generate more leads in an enterprise market, how to stimulate demand for premium features, user-friendlyness and the importance of being first-to-market. Given its relevance to this thesis' central themes it is discussed in Chapter 4.