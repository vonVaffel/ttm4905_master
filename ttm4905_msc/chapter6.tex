\chapter{Proposal of Business Model}
This chapter will present the reader with the freemium-based business model proposal for MazeMap. It will be proposed based on findings and results from other parts of this thesis, particularly from Chapter 4 \& 5. The Business Model Canvas described in Appendix C will be used as a fundamental framework to describe and realise the model.


\section{The Freemium Indoor Map Service Business Model}
\subsection{Customer Segments}
While this thesis has primarily been focused on one customer segment, namely \glspl{hei}, an \gls{ims} provider such as \gls{mm} may focus on additional customer segments: Hospitals, venues and shopping malls (along with \glspl{hei}) constitutes \gls{mm}'s self-described customer base~\cite{mazemap}. Furthermore, more specific and niche markets can be described in cruise ships, sporting arenas (football stadiums, car-racing arenas etc.), airports and museums. The actual procurers of MazeMap are senior personnel within any aforementioned institution or organisation, that govern the procurement activities. Focusing on the customer segment presented in this thesis, a divide can be said to exist between publicly owned \glspl{hei} and privately owned \glspl{hei}. These may have different emphasis on price in long or short terms. Public institutions may be more concerned with total costs and the quality, ease-of-use and usefulness of a service, while private institutions may be more sensitive to price in order to satisfy shareholders by keeping procurement costs down. All \glspl{hei} contacted were selected on a criterion of being of a certain size in terms of staff and enrolment. In the survey, Q6 indicated that "demand" was a possible deterrent from an \gls{ims}, and this option was more than often selected by \glspl{hei} smaller in size compared to the rest of the respondents. As such, this should be taken into account when targeting \glspl{hei} as a customer segment.


From the customer's perspective, a low amount of financial means is needed in order to obtain \gls{mm}'s \gls{ims}. Having existing infrastructure such as Wi-Fi for implementing navigation features, might be seen as a requisite from the customer's perspective, should they desire navigational features. Given the characteristics of \gls{b2bc}, any procurers of an \gls{ims} should have an already established consumer base, with students and staff constituting \glspl{hei}' customer base. To sum up:


\begin{itemize}
    \item Procurers of MazeMap: Someone who governs procurement activities
    \item Freemium might be harder to implement due to regulatory affairs
    \item What needs to be in place customer side: Low amount of financial means, optionally Wi-Fi for navigation and an existing base of clients
\end{itemize}

\subsection{Value Propositions}
How a service generates value is instrumental in choosing target customer bases. \glspl{hei} are able to benefit from an \gls{ims} such as \gls{mm} for a plethora of reasons: As explored in International Business Potential for Analytics of Room Utilisation~\cite{karlbernhoffbinde2015}, by increasing room utilisation through analytics of how space is being utilised leads to several potential benefits: Increasing enrolment, reducing maintenance cost, alleviating environmental stress and reducing opportunity costs. \glspl{hei} experience an influx of new students and staff at the beginning of semesters, and an \gls{ims} can save both time and frustration among these in relation of finding their way around campus. Additionally, visitors are able to streamline their visiting experience by having an \gls{ims} provide parking instructions. Furthermore, \glspl{hei} are often the site of conferences consisting of first-time visitors who also can benefit greatly from finding out where to go and at what time, given \gls{mm}'s integrability with time-table and scheduling systems. Among both external and internal maintenance personnel, an \gls{ims} enable less time being spent on finding the correct place to be and lets personnel do their tasks more effectively, while greatly reducing the need for new staff to be shown around. 


With interactive maps allowing both customised and automatic generation of meta-data, an \gls{ims} can provide its users with fleet management. This is related to how users can find and locate equipment, both emergency-related equipment such as fire extinguishers and non-critical equipment such as printers. In this context, the emergence of the Internet of Things is also applicable, given that a lot of previously autonomous systems are now interconnected. An \gls{ims} enables tracking of these assets, and can satisfy the demand for making these things visible to its users.  Another important use-case is how a fire department can save time in critical moments by using an \gls{ims} to discover and locate fires or other accidents, as opposed to decoding messages from a fire alarm system. In the context of another customer segment not explicitly discussed here (hospitals), it is possible to save costs by reducing the number of missed appointments~\cite{mazemaphosp}, by employing an \gls{ims}. This can also be applied to \glspl{hei} albeit at possibly smaller cost savings, due to \glspl{hei} not being as appointment-based as hospitals.  


In terms of freemium, there are obvious cost savings for procurers of freemium-based \glspl{ims}. As the \gls{b2b} sales pipeline is longer than in a \gls{b2c} setting, a try-before-you-buy product may prevent compounded costs as a result of a long sales process. As was stated by the respondents of the survey, price presented an entry barrier to the potential purchase of an \gls{ims}, and there were some willingness to pay for some additional services. By taking this into account, this bodes well for implementing a freemium-based \gls{ims}.
To summarise:

\begin{itemize}
    \item Digital interactive maps
    \item Optimise utilisation through analytics
    \item Directions for meeting-place and nearest parking for visitors
    \item Reducing uncertainty and stress for new students and staff
    \item Maintenance personnel: Professions that see frequent changes of staff and working locations. Indoor maps enable these to find their way quicker, saving personnel costs
    \item Fleet management, equipment tracking and the Internet of things
    \item Alarms shows up on maps rather than a code for a specific location. Can save time in emergency situations
    \item Reduce missed appointments
    \item Provide places of interests visually as opposed to textually
    \item For freemium: Cost savings by being free initially
    \item For freemium: Try before you buy, no big initial investment. Purchase according to need
    \item Further value-adding services
\end{itemize}

\subsection{Channels}
In order to be able to deliver its value propositions to the customer, a company delivering an \gls{ims} must use appropriate sales channels. Given that indoor maps is still a relatively new service, determining what channels to use can be difficult. It should be stressed that given the survey's respondents unfamiliarity with indoor maps, that such a service needs to emphasise its potential through value propositions, and bring awareness around them. Being an early mover proved to be important in the case of Teleopti discussed in Chapter 4, and given the fact that only 35\% of respondents already employed an \gls{ims}, further validates this notion. 


In terms of accelerating customer acquisition, the direct approach often taken in other \gls{b2b} scenarios may prove to be counter-productive for this purpose. First, the value proposals have to be brought forth to decision makers at an \gls{hei}. Then a top-down or a down-top recommendation may occur: If senior personnel take a liking to the service, their authority on the matter may influence its users (which are the consumers in the \gls{b2bc} setting) to start using a service. This is the top-down approach. The down-top approach occurs whenever demand for such a service starts with the consumers either through word-of-mouth or by users who have seen an \gls{ims} in action at another institution, similar to Box. This may in turn influence decision makers by wanting them to satisfy their respective consumers. As an \gls{ims} obtains more customers, the latter approach becomes more attractive, as its user base expands creating a demand that was scarce in the first place. The former approach can also greatly benefit from this through cross-references such as when success stories between \glspl{hei} are shared. This of course can also generate bad-will, should the service provided turn out to be lacking. During the initial roll-out phase of a freemium-based \gls{ims}, direct contact may be an important channel, however, as customer acquisition and lead generation become more important factors, direct contact becomes too time consuming and should be reserved customers who represent a larger business potential. As suggested by Lamminpää in Chapter 4, being visible at trade conventions may be a good tool for leads generation by direct contact. As opposed to the previously mentioned outreach-based direct contact, being visible at events hosted by organisations such as \gls{scup}, \gls{tefma}, \gls{cele} and \gls{appa}, may prove to be a more effective way of acquiring new customers from the \gls{hei} customer segment. Lastly, this channel type handles almost all of the product phases mentioned in Appendix C, barring after-sales.


MazeMap's webpage may also be used increasingly as a sales channel in the case of delivering a freemium-based service. As discussed in Chapter 4, entry barriers should be kept as low as possible. If the webpage is used as a sales channel it needs to provide adequate information regarding the value propositions offered to enable users to evaluate these propositions. Additionally, a service such as Google AdWords as employed by Limecraft, may also strengthen the webpage as a sales channel, by bringing awareness around the service provided. Given the \gls{saas}-nature of the product, the webpage can be used as a vessel to deliver the product to customers, and while users in the freemium paradigm do not purchase anything in the traditional sense of the word, the purchasing part could be handled by requiring users to register as the only barrier for using the service. To go with this, thorough and clear documentation should also be provided for these customers. Any premium parts of the service could be greyed out, but made readily purchasable depending on the service. 


Through \gls{mm}'s partnership with Cisco, the Cisco marketplace is another available sales channel~\cite{ciscomarket}. This platform provided by a well-known actor such as Cisco presents great publicity through the its brand, which provides services that are likely to meet any security concerns. Cisco is also a licensed, worldwide reseller, which may yield great value as a sales channel. Below is a summary of the various channels:


\begin{itemize}
    \item Direct approach
    \item Develop leads among potential customers
    \item Webpage
    \item Cisco marketplace
    \item For freemium: Hard to argue against initial high costs as with a non-freemium system
\end{itemize}

\subsection{Customer Relationships}
The need for establishing good and reasonable customer relationships is great, and in the case of freemium, generating leads is the most desirable outcome. These relationships changes as the product goes through different phases. Table~\ref{classiccr} shows a possible strategy for customer relationships under a non-freemium model, which focuses mostly on personal relationships. In a freemium model with a rapidly accelerated customer acquisition process, this can possibly lead to increased cost in staffing, sunk costs and costs resulting from failed sales. Therefore a more automated customer relationship model must be presented, seen in Table~\ref{relationshipstrat}. This attempts to combat the need for staffing dedicated to different parts of the sales cycle, by moving towards an autonomous, self-servicing system as the service matures. By providing rigorous documentation and self-help tools for the non-paying customers, as well as reaping potential benefits from communities around the product getting traction, this should be readily implemented. Lastly, as the service grows in users the number of bugs may increase, and as such, patches that are platform-wide and payment model-agnostic should be in place to maintain customer satisfaction.


\begin{table}[]
\centering
\caption{Customer relationships for different product phases, classic model~\cite{karlbernhoffbinde2015}}
\label{classiccr}
\begin{tabular}{|l|l|}
\hline
\textbf{Phase} & \textbf{Customer Relationship Strategy} \\ \hline
Pilot          & Co-creation, key partnerships           \\ \hline
Introduction   & Dedicated Personal Assistance           \\ \hline
Growth         & Personal Assistance                     \\ \hline
Maturity       & Communities                             \\ \hline
Decline        & Self Service                            \\ \hline
\end{tabular}
\end{table}


\begin{table}[]
\centering
\caption{Customer relationships for different product phases, freemium model}
\label{relationshipstrat}
\begin{tabular}{|l|l|}
\hline
\textbf{Phase} & \textbf{Customer Relationship Strategy}    \\ \hline
Pilot          & Dedicated personal assistance, co-creation \\ \hline
Introduction   & Self-service, personal assistance          \\ \hline
Growth         & Automated services                         \\ \hline
Maturity       & Communities                                \\ \hline
Decline        & Self-service                               \\ \hline
\end{tabular}
\end{table}

Emphasis should also be put on the creation of communities and cross-reference sales, hereunder inter-customer interactions. This to enable users to share knowledge and develop good usage practises, alleviating some of the need for customer service, should the service expand with the freemium model applied. The nature of the service does not induce competition among customers, and as such there is little to no point in not sharing the experiences of an \gls{ims}. On the contrary, the value and performance of the service may increase for all parties involved, which can only be viewed in a positive way. 


With freemium comes additional freedom in how a product is bundled. A solution like \gls{mm}'s can easily be modularised, with more full-fledged customer support, guidance or even \gls{crm} as premium features. By monetising this part of the service at a mature product stage, means that any early movers will not feel deterred from obtaining this type of service. Smaller and medium sized venues who may have less requirements from an \gls{ims} would ideally not need support if the self-support tools are sufficient in themselves, and larger institutions may be the target customer segment for more extensive and paid support features. Below is a summary of the proposed customer relationships:

\begin{itemize}
    \item As service matures, less active relationships, more automation
    \item Interactions between the businesses under the B2B\&C paradigm, sharing of experiences
    \item For freemium: Guidance as a paid service
\end{itemize}

\subsection{Revenue Streams}
Given that \glspl{ims} are still in its infancy product phase-wise, establishing that the services provided can be value-adding is important. Additionally, even though no monetary investment is needed upfront from the customer's perspective, the survey indicated that there are other factors such as demand and security concerns that may hinder the procurement of an \gls{ims}. It must therefore be communicated extensively that those factors should be of no concern by, for instance, pointing out the partnership with Cisco for security concerns and clearly communicating the value proposals in order to create demand. The future might bring additional need for services hitherto unknown, and these should be included as part of the \gls{mm} product range, to create a lock-in effect. 


In choosing which parts to be free and which ones that should be monetised, several considerations must be made: Create too few free modules and the service provided might be lacking for its purposes and create no demand. Creating too many free modules will risk cannibalisation the low end of the market~\cite{keller2011strategic} and generate little to no revenue. Considering these options, it might be better to initially offer too few modules as \gls{mm} already has an established customer base. While the survey showed that the lack of demand of an \gls{ims} is present, it could be risky to dedicate too much resources in offering non-paying customers modules or features should otherwise be monetised. 


Several parts of \gls{mm}'s product exists, which can be decomposed into different modules described in Table~\ref{modules}. For applying a successful freemium strategy, the choice of module(s) to include for free is crucial. The free parts should include the bare minimum for an \gls{ims}: An indoor map, basic map editor and an application to view the indoor map. Any additional services should be monetised. There are several ways to monetise the modules not included in the free package and as such two different payment models can be presented:


\begin{table}[]
\centering
\caption{Proposed modularisation of MazeMap}
\label{modules}
\resizebox{\textwidth}{!}{%
\begin{tabular}{|l|l|l|}
\hline
\textbf{Indoor maps}                             & \textbf{Integration} & \textbf{Navigation} \\ \hline
Map editor: basic                                & SMS notification     & Indoor positioning  \\ \hline
Map editor: advanced                             & Room reservation     & Indoor pathfinding  \\ \hline
Interactive maps (browser only)                  & Applications         &                     \\ \hline
Interactive maps (any platform)                  & IT-systems           &                     \\ \hline
Facility management (automatic updating of maps) & Timetables           &                     \\ \hline
Analytics                                        & API/SDK              &                     \\ \hline
Automatic updating of maps                       &                      &                     \\ \hline
\end{tabular}%
}
\end{table}

\subsection{Subscription}
As with Box (presented in Chapter 4) a subscription model for the premium, value-adding services may be implemented. For services such as analytics and automatic updating of maps and metadata a recurring payment model makes sense, given that they depend on several variables that may change customer-side. Other premium modules such as advanced map editors, platform independence and integration may not be as applicable to a subscription based payment model. However, there is nothing directly speaking against this way of monetising a service. Given that \gls{mm} delivers a \gls{saas}, operational costs must be taken into account, and subscription based features may cover these costs accordingly. After registering and uploading floor plans, users should be presented with a dashboard of the services available to them. The additional premium modules may be greyed out and activated when desired. This will then in turn induce either monthly or yearly recurring payments, depending on the type of service. Given that buildings change more slowly than points of interest, different rates of recurrent payments should be presented depending on the module. Additionally, usage quotas may be introduced. This works slightly different than the previously mentioned model; some modules are free to use and try out. The difference between the free and premium in this instance, is that free users are able to use premium features, but only up to a certain cap. This cap may let the premium features be time-limited or data-volume-limited, with restrictions removed upon setting up a payment plan.  

\subsection{One-time payments}
More in line with traditional, non-recurring payments, several of the modules fit better into this category, as opposed to the subscription model. Under this model customers would be presented with the premium modules greyed out, and pricing info on the respective models. Customers wanting to activate a particular module for their institution may do so after paying a one-time fee for said module. Under this model, the aforementioned time or data volume limitations on premium features may be imposed as an alternative way of trying out the various services. It should be noted that combining these two models is also a possibility, by having some modules appropriately subscription-based and some one-time-fee-based.


Free trials of the different premium modules can also be offered to be more in line with the try-before-you buy sentiment, central to the freemium business model. Additionally, some features can be free on rotation, for instance from one month to the next, which may increase the interest in some of the less popular features. This also opens up the possibility of evaluating the premium features available, by adjusting the price for less popular modules, while increasing price for the modules in demand. It is also possible to bundle some of the premium features, as a technique to reduce customer's ability to evaluate and reserve themselves at given price points. This provides a lock-in effect among customers, and may reduce the starting costs associated with enabling some of the premium services. To sum up:

\begin{itemize}
    \item Concept of freemium: The few pays for the many
    \item Subscription based premium modules
    \item One-time-fee premium modules
    \item Bundling
\end{itemize}

\subsection{Key Activities}
Along with the key resources, the key activities form the way value propositions are delivered to the customers. Given the \gls{saas} nature of the product delivered, software development can be listed as one key activity, which relates to the ongoing development and maintenance of the key resource that is \gls{mm}'s platform. While operating within a freemium business model, a more secluded approach might be taken in terms of business development. Targeted marketing and sales should be reserved to larger customers, since one of the main goals of freemium is to accelerate customer acquisition. On that note, it is important for a supplier of an \gls{ims} not to completely forego this type of customer under the freemium paradigm, as that will leave the larger enterprise market largely untouched. When a customer has registered in order to use the free parts of the \gls{ims}, it is important to provide sufficient support and documentation, in order for customers to get up an running, thus retaining the customer base. Customers will quickly lose interest in a new type of service provided such as an \gls{ims}, if adequate introduction to the product is absent. As such, providing this can be seen as a key activity. 


Roughly 80\% of \gls{mm}'s user-base don't use any form of positioning services, so customers wanting the indoor navigation and pathfinding modules may be directed towards the solution provided by \gls{mm}'s partner Cisco for implementation of the Wi-Fi based navigation feature. There are however, a plethora of emerging technologies that provide this, some with far greater accuracy than what is possible with Wi-Fi: iBeacons (\gls{ble}-based, signals between smartphone and beacon)~\cite{ibeacon2016}, IndoorAtlas' IPS system (geomagnetic indoor positioning, smartphone with magnetic sensors reacting to earth's magnetic fields)~\cite{indooratlas2016} and Sensor Fusion by indoo.rs (using a smartphone's built in sensors)~\cite{indoo.rs2016} to name a few. \gls{mm} has stated that supporting more positioning and navigation features will be made possible should there be a demand for them in the future (Jelle, personal communication 17.09.2015). Integrating maps into a customer's existing IT-systems and maintaining the map data and metadata can also be viewed as a key activity, providing a certain lock-in effect among customers. 


After purchases, emphasis should be put on maintaining the platform ensuring that a good and stable service is provided to the customers. The premium modules should act as means of revenue, and together with maintaining the platform it forms an important key activity, as it is vital for customer retention and may hamper the creation of communities. Enabling community creation post-sales should also be viewed as being paramount. This can be done through official forums, where users share both share knowledge and solves problems together, as may be necessary in the event that the customer base expands rapidly. Again, this to minimise supporting costs, however some customers with more advanced needs who operate on a larger scale may need dedicated personal assistance. In turn, this customer group is also expected to pay for more premium features, possibly covering the support cost margin.

\begin{itemize}
    \item For freemium: Providing adequate level of support, particularly during the implementation phase
    \item Key Account management and business development - large customers with many needs
    \item Indoor mapping constitutes roughly 80\% of usage
    \item Accommodating for indoor navigation and pathfinding via Cisco and other partners
    \item Integration of maps
    \item Reworking and updating map data and metadata
    \item Integration into booking systems
    \item Marketing, sales and customer support
    \item After purchase: Revenue through premium modules
    \item After purchase: Enabling community creation 
\end{itemize}

\subsection{Key Resources}
Since the method of delivery of \gls{mm} is \gls{saas}, several resources needs to be allocated in order to provide and create the services needed to fulfil the value propositions. \gls{mm}'s servers containing the service provided, is a hugely important asset in this case. Without these, the service would not be able to be offered as a \gls{saas}. Having a centralised, consolidated infrastructure enables faster bug-fixing, and will provide customers with an up-to-date version of the services at all times. Regarding freemium, the most important aspect of the key resources is the engine that converts floor plans to interactive maps. This also makes it possible to lower the price point drastically, and through this gaining a competitive advantage over competing businesses, as discussed in Appendix C. In a scenario where freemium is implemented, the rapid onset of an expanding customer base would simply be infeasible if a manual process was in place to make the interactive indoor maps. With a robust engine generating maps, little has do be done from \gls{mm}'s side per new customer.


In regards to human resources, the software developers with expertise and deep knowledge about the technical aspects of the \gls{ims} solution can be seen as a vital key resource. Given that an \gls{ims} often have to interface with other existing systems like timetable,room reservation and IT-systems in general, it is imperative that competent staff is in place in order to satisfy the needs of the customers. Furthermore, personnel working in R\&D and marketing may be easily overlooked in this situation, but these groups provide vital services in obtaining new customers (an important part of the freemium paradigm), and expands the value offerings towards the customers by meeting or creating demand. Having a partner in Cisco providing a distribution channel for the product, can also be seen as a key resource. Additionally having a close relationship with early movers such as \gls{ntnu}, can function as a test-bed for new features, as well as being a point of potential success-stories. There is also a benefit of having \glspl{hei} as a customer, namely enabling recruitment to further strengthen the workforce. A categorised overview of the most important key resources can be seen in Table~\ref{keyresources}, and a summary follows below:

\begin{itemize}
    \item Servers
    \item The floor-plan-to-interactive-map engine
    \item Software developers
    \item Partners
    \item Customer's mapping data + metadata
\end{itemize}


\begin{table}[]
\centering
\caption{Key resources by category}
\label{keyresources}
\begin{tabular}{|l|l|}
\hline
\textbf{Category} & \textbf{Resources}                                                                                                                        \\ \hline
Physical          & Servers                                                                                                                                   \\ \hline
Human             & R\&D, marketing and software development personnel                                                                                        \\ \hline
Financial         & \begin{tabular}[c]{@{}l@{}}Customer acquisition through the free part of the product, \\ and revenue through premium modules\end{tabular} \\ \hline
Intellectual      & Floor-plans-to-interactive-maps engine and partnerships                                                                                   \\ \hline
\end{tabular}
\end{table}

\subsection{Key Partnerships}
In the previous sections the partnership with Cisco has been affirmed, and can therefore be viewed as a key partnership. Having Cisco as a distribution channel can be a tremendous asset in a freemium model, as Cisco's existing customer relationships may be more easily transferable over to \gls{mm} if a service is offered through a freemium model. Furthermore, given the Cisco's leading position and brand name, this also alleviates some of the risks potential customers may associate with an \gls{ims}.  As the customer base expands, further partnerships with other actors on the market is also possible, however, this may deteriorate an already existing relationship, should Cisco's competitors form partnerships with \gls{mm}. 


Unless customers are reserved or exempt from it, gathering usage- and metadata may also be the backbone of new partnerships being made. In this bi-lateral relationship between customer and company, mutual benefits may be obtained on both sides: Customers are able to receive a more tailored product according to their needs, increasing the perceived and actual exclusiveness of the service provided and companies are able to further evaluate and assess the service provided based on operational intelligence gathered from the customers. Several respondents of the survey cited security as a concern in the context of potential procurement, making trust between customer and company an issue that should be addressed by reassurance and appropriate usage of the data provided by the customer. As the customer base expands to customers worldwide, partnerships with companies providing redundancy nodes might be necessary. This is entirely optional, as these services constitute a separate \gls{b2b} transaction between the \gls{ims} provider and the provider of redundancy nodes. However, given the nature of an online-based \gls{saas} it is important to provide a fast and reactive \gls{ims} in order to maintain good relationships to customers. Redundancy nodes may improve this performance for customers far away from the servers of the \gls{ims}. Summed up, the takeaways are as follows: 


\begin{itemize}
    \item Positional services through Cisco
    \item Distribution channels in Cisco
    \item Operational intelligence
    \item Cloud storage: servers and redundancy nodes for better access throughout the world.
\end{itemize}

\subsection{Cost Structure}
The last building block of the business model is also one of the most vital parts, as it is detrimental for operating a successful business. This pertains to which key resources and key activities that drive the costs. An \gls{ims} such as \gls{mm} is primarily a value-driven venture, as most of the activities concerns the creation of value among its customers in addition to counselling. However, in the case of freemium, the venture can stray further away from a value-driven business towards a more cost-driven one. In the startup phase and during development of key value propositions such as the mapping engine, the costs will be relatively high, but the business will in time benefit from economies of scale. These economies may be even larger if freemium is implemented, as expanding the customer base could achieve exactly this, and great returns in investments may be had. In general, \gls{saas}-like services will in most cases be an expensive venture, and apart from initial costs, the running cost of maintaining a highly qualified staff consisting of R\&D, marketing and software personnel can be expensive. However, given that the customer segment of \glspl{hei} also can act as a recruiting arena, these recruitment costs may be kept low. 


Given that each customer presents a liability in terms of costs, it is important to keep marginal costs as low as possible, with this pertaining to the freemium model in particular. Some of these costs include integration costs, customisation of maps and interfacing with existing infrastructure should the need present itself. A way to keep these costs low, lies in benefiting from the scale of operation that freemium as a business model might bring. If several institutions use similar interfaces for the premium one-time-fee based services, the marginal work needed and costs of this is lowered as the customer base expands. The survey results indicated a general interest in indoor maps, and particularly a freemium based \gls{ims} among the respondents, and as such it would be expected that the aforementioned benefits of scale would come into fruition.


Not to be overlooked is the cost of renting or purchasing facilities such as offices or support centres. This cost will increase as the size of the operation expands with offices in more countries and more customers, which may present a larger demand for supporting services. The latter can be somewhat mitigated through monetising advisory and extensive supportive services, thus providing this as a premium service. If sufficient means is put towards making thorough and easily understood documentation of the service provided, this cost can be lowered even further. Lastly, after the initial setup phase the running costs should not be of any large magnitude. Even if the economies of scale may induce higher costs, the premium features may be adjusted in price or functionality in order to cover the costs related to expanding the customer base. Due to the scalable map-generating platform and the fact that customers are largely able to administer their own maps as they see fit, these processes would most likely be inexpensive. An overview of the costs is shown in Table~\ref{costs}, and a summary follows below:


\begin{itemize}
    \item Value driven cost structure: value creation and counselling
    \item High startup costs: Software development, R\&D and marketing team, servers and offices
    \item Offices/personnel: Development and renting facilities.
    \item Low running costs: Scalable platform, customer is able to administrate maps themselves (can add POIs etc.)
\end{itemize}


\begin{table}[]
\centering
\caption{Overview of costs}
\label{costs}
\begin{tabular}{l|l|l|}
\cline{2-3}
\textbf{}                                     & \textbf{High}                                                                                          & \textbf{Low}                                                                                                    \\ \hline
\multicolumn{1}{|l|}{\textbf{Fixed costs}}    & \begin{tabular}[c]{@{}l@{}}Worker Salaries\\ Software development\end{tabular}                         & \begin{tabular}[c]{@{}l@{}}Costs of operation\\ Rental of facilities\\ Servers \& redundancy nodes\end{tabular} \\ \hline
\multicolumn{1}{|l|}{\textbf{Variable costs}} & \begin{tabular}[c]{@{}l@{}}Deployment of services\\ Counselling\\ Integration of services\end{tabular} & \begin{tabular}[c]{@{}l@{}}Hardware\\ Marketing\end{tabular}                                                    \\ \hline
\end{tabular}
\end{table}

\section{Summary of the Business Model Canvas}
Figure~\ref{canvas1} shows a visual representation of the business model, generated in the Business Model Canvas-framework:

\begin{figure}
    \centering
    \includegraphics[width=\textwidth]{figs/busmod.png}
    \caption{Proposed freemium-based business model for an IMS such as MazeMap, in the Business Model Canvas-framewok}
    \label{canvas1}
\end{figure}

