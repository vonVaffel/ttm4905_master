\chapter{An Introduction to the Business Model Canvas Framework}

This chapter introduces the reader to the framework that is the Business Model Canvas, used to describe the proposed business model in Chapter 6. This particular framework has its merits in that it is used to invent, challenge and describe a business model \cite{strategyzer2016}. Further advantages include that it strips away any superfluous elements, improving its readability and enables users and business owners to focus on the most vital elements of a business model. It is also immensely flexible, as changes can be readily made without changing everything thanks to its modular design \cite{osterwalder2013business}. 


\begin{table}[H]
\centering
\caption{Building blocks of the Business Model Canvas}
\label{tab:canvas}
\begin{tabular}{|l|l|}
\hline
\textbf{Product}                                    & Value proposition     \\ \hline
\multirow{3}{*}{\textbf{Customer interface}}        & Customer segments     \\ \cline{2-2} 
                                                    & Relationship          \\ \cline{2-2} 
                                                    & Distribution channels \\ \hline
\multirow{2}{*}{\textbf{Financial aspects}}         & Cost structure        \\ \cline{2-2} 
                                                    & Revenue stream        \\ \hline
\multirow{3}{*}{\textbf{Infrastructure management}} & Key partnerships      \\ \cline{2-2} 
                                                    & Key activities        \\ \cline{2-2} 
                                                    & Key resources         \\ \hline
\end{tabular}
\end{table}

The framework consists of nine building blocks with varying degrees of importance, but an important relationship between them. When all building blocks are determined, they make up the total strategy of how a business would operate, make money and if desired, capture market shares. Table~\ref{tab:canvas} shows the different building blocks by their respective categories, and Figure~\ref{fig:businessmodelcanvas} shows the empty template for setting up the business model. Below follows a step-by-step description of the respective building blocks.


\section{Customer segments}
This building block consists of the different customer segments a company wishes to concentrate on and reach out to. A market may be single or multi-sided, containing at a minimum a segment per market side. Media companies, credit card companies and to some extent social networking sites among others, fall into a multi-sided market category. It is important to note that when offering a value proposition to a market, market size is an important factor, as smaller, niche markets may be a viable market segment as opposed to a bigger market. Finally, diversified products offered through the value proposition may be used to reach smaller subsets of the market.

\section{Value proposition}
The value proposition is the building block that describes how a company is different from other competing companies. It details how a company's products or services can create value for their respective customer segments, where values may be qualitative or quantitative. Several means of creating value among customers  includes lower pricing (price-sensitive segments), design (aesthetically appealing products), status (well-known brands), customisation (tailor-made services or products) and performance-driven products. Additionally, introducing a brand new disruptive technology i.e. cell-phones, may benefit a business' customer segment even though it may initially be viewed as unnecessary. Offering consultancy services can also be a value proposition in itself, for instance IT-consultancy services. Concretely, in the framework one wishes to map a value proposition to a particular customer segment, in order to identify the needs of this particular segment.

\section{Distribution channels}
Distribution channels concerns how a value proposition is delivered to a customer segment. A channel serves multiple purposes, with the most important being raising awareness around the products or services a company is offering and aiding customers in understanding the value propositions. It also serves a perhaps equally important function in allowing for products or services to be purchased. Furthermore, the channels can be broken into different phases, depending on what phase a product is in. These include:

\begin{itemize}
    \item \textbf{Awareness}: Raising awareness among customers.
    \item \textbf{Evaluation}: How customers are able to evaluate the value proposition.
    \item \textbf{Purchase}: How products are purchasable from the customer's point of view.
    \item \textbf{Delivery}: How a value proposition is delivered to a customer.
    \item \textbf{After sales}: How on-going customer support is handled post-purchase.
\end{itemize}
Along with the next block (customer relationships) the channels block forms how a business interfaces with their customers. 

\section{Customer relationships}
The different types of relationships a business establishes with their respective customer segments are detailed in the customer relationships block. From a business' perspective, having good customer relationships may entail several benefits in increasing their sales volume, gaining new customers and keeping their existing customers from leaving. The perhaps simplest relationship between a business and a customer lies in personal assistance, where actual, dedicated personnel from the business side is serving any customer need from any part of the sales cycle. In a \gls{b2b} market one may have one dedicated person or team per customer, while in a \gls{b2c} market this might not be manageable, and call centres or e-mail respondents serves the purpose better. On the flipside, having customers manage themselves entirely either through robust online self-services or inter-customer relationships is also an option depending on the product or service provided. Lastly, co-creation where companies and customers share responsibility for the product is a modern take on a customer relationship, seen in social networks and user-creation-oriented services such as YouTube.

\section{Revenue streams}
How much cash-flow each customer segment generates constitutes the revenue stream building block. It is essential for the profitability of a product or service, with revenue streams being either a one-time fee or recurring payments. Several ways to generate revenue streams include:

\begin{itemize}
    \item \textbf{Subscription fees: }By selling a service, a business may charge its customers of that service for any given time period. 
    \item \textbf{Licensing: }In companies where some form of intellectual property is made, it is possible to generate revenue by the sales or lending of these properties.
    \item \textbf{Advertising: }This type of revenue stream is generated from advertising a product or service on the behalf of some other business entity.
    \item \textbf{Brokerage fees: }By providing services between two parties and taking a fee for the transactions that take place.
    \item \textbf{Asset sale: }Selling the rights to one instance of a product falls into this category, and is the most traditional way of exchanging goods.
\end{itemize}

At this point of setting the model up, the customer segments are linked to its respective value propositions. Each of these should at this point be linked with a revenue stream.

\section{Key activities}
The key activities are the detrimental matters that a business needs to attend in order to deliver its value propositions. Together with key resources they are vital in creating and offering the value proposition to the customers, earning revenues and keeping customers satisfied. Consultancy and service-oriented businesses often revolve around problem solving as a key activity, helping others with new and existing problems. For manufacturing companies, proposing, making and delivering products is a key activity, while software- and banking-service companies may have a robust platform they offer their customers. In the case of the latter, the platform itself is the main component of their key activities. Lastly, it is important to connect the key activities to the value propositions, as the key activities are the main drivers of the value propositions.

\section{Key resources}
The key resources in this framework is absolutely vital for businesses in order to provide and create value propositions for its customers. Together with the key activities, they enable the generation of revenues, maintaining customer relationships and reach markets. Types of key resources include physical (buildings, manufacturing plants), human (consultancy and r\&d services), financial (gaining and edge on competitors by lowering the price point) and intellectual (intellectual properties, brands etc.). The goal 
of the key resources is for the business to surpass competitors on key areas of the key resources.

\section{Key partnerships}
This building block concerns the partnerships, suppliers and other third party entities needed to make the business model work. The partnerships can vary in nature, where competitors and non-competitors can forge alliances or creating joint ventures. Any external solutions have to be supplied by third parties be it manufacturing parts or personnel. Motivations for creating partnerships include optimising the allocation of resources, reducing risk and to cut down on activities not vital in delivering a final product or service. As such, key partnerships can be linked with activities that aren't necessarily key to drive a business' value proposition. 

\section{Cost structure}
Different cost structures include fixed costs, variable costs, economies of scope and economies of scale. Fixed costs are volume-independent, and are usually exemplified by employee salary, rental and other facility costs. Variable costs are volume-sensitive, while economies of scale concerns the costs changing as a result of a change in the scale of operation. Economies of scope on the other hand benefits businesses that diversify the number of different products or services offered, and is volume-insensitive in this regard. Different cost-structures exist in cost-driven and value-driven models. The former focuses on minimising the costs, and the latter pertains to companies that are less concerned with price and focuses more on value creation.
Lastly, it is important to note the relationship between cost structures and key activities, as the key activities drive a business' cost structures. 

\begin{figure}[]
    \centering
    \includegraphics[width=10cm]{figs/busmodcanvas.png}
    \caption{Business Model Canvas template. \textcopyright businessmodelgeneration.com}
    \label{fig:businessmodelcanvas}
\end{figure}